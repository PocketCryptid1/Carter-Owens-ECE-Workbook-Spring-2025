\section{VERIFICATION OF REQUIREMENTS}

Possible verification methods include:

\bigskip

\begin{enumerate}

    \item \textbf{Inspection:} \\
    Inspection is a method of verification consisting of investigation, 
    without the use of special laboratory appliances or procedures, to 
    determine compliance with requirements. Inspection is generally 
    nondestructive and includes (but is not limited to) visual examination, 
    manipulation, gauging, and measurement.

    \item \textbf{Demonstration:} \\
    Demonstration is a method of verification limited to readily 
    observable functional operation to determine compliance with 
    requirements. This method does not require special equipment 
    or sophisticated instrumentation.

    \item \textbf{Analysis:} \\
    Analysis is a method of verification taking the form of processing 
    accumulated results and conclusions to provide proof that 
    verification has been accomplished. Analytical results may be based 
    on engineering study, interpretation of existing information, 
    similarity to previously verified requirements, or results from 
    lower-level tests or demonstrations.

    \item \textbf{Direct Test:} \\
    Test is a method of verification employing technical means, including (but not 
    limited to) the evaluation of functional characteristics using 
    instrumentation, test equipment, simulation tools, and established 
    test procedures to determine compliance with requirements.

\end{enumerate}

\bigskip

% --------------------------------------------------
% Verification Coverage Matrix
% --------------------------------------------------

\subsection{Verification of Engineering Requirements}

Table~\ref{tab:verification} identifies the verification method used for each Engineering Requirement defined in this specification.

\begin{longtable}{|c|c|p{7cm}|}
\caption{Verification Method for Engineering Requirements} \\
\hline
\textbf{Requirement No.} & \textbf{Verification Method} & \textbf{Verification Description} \\
\hline
\endfirsthead

\hline
\textbf{Requirement No.} & \textbf{Verification Method} & \textbf{Verification Description} \\
\hline
\endhead

ER-1 & Inspection & Measure dimensions and mass of VTX module to confirm compliance. \\ \hline
ER-2 & Inspection & Measure dimensions and mass of VRX module to confirm compliance. \\ \hline
ER-3 & Analysis & Confirm component ratings and environmental specifications… \\ \hline
ER-4 & Test & Perform vibration and impact tests… \\ \hline
ER-5 & Test & Conduct end-to-end latency measurement… \\ \hline
ER-6 & Demonstration & Verify HDMI output resolution on external display. \\ \hline
ER-7 & Inspection & Confirm RF connector type. \\ \hline
ER-8 & Analysis & Verify regulator design supports drone battery input. \\ \hline
ER-9 & Demonstration & Demonstrate VRX operation on AC adapter or power pack. \\ \hline
ER-10 & Demonstration & Demonstrate mode switching during operation. \\ \hline
ER-11 & Test & Perform RF loss-of-link test and measure reconnection time. \\ \hline
ER-12 & Demonstration & Verify MSP communication with flight controller. \\ \hline
ER-13 & Analysis/Test & Ensure RF emission compliance with FCC Part 15. \\ \hline
ER-14 & Demonstration/Test & Verify uplink ACK and telemetry performance. \\ \hline

\end{longtable}


\bigskip

% --------------------------------------------------
% Stakeholder Requirements Verification Table
% --------------------------------------------------

\subsection{Verify Coverage of Stakeholder Requirements}

\begin{table}[h]
\centering
\begin{tabular}{|c|c|C{6cm}|c|c|}
\hline
\textbf{Paragraph Number} & \textbf{Test Type} & 
\textbf{Tester’s Name} & \textbf{Pass/Fail} & \textbf{Date} \\
\hline
 & & & & \\
\hline
 & & & & \\
\hline
 & & & & \\
\hline
 & & & & \\
\hline
 & & & & \\
\hline
 & & & & \\
\hline
 & & & & \\
\hline
 & & & & \\
\hline
 & & & & \\
\hline
\end{tabular}
\end{table}
